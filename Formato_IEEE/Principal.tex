\documentclass[12pt]{IEEEtran}
\usepackage{amsmath}
\usepackage{mathptmx}
\usepackage{braket}
\usepackage{amsfonts}
\usepackage{amssymb}
\usepackage{amsthm}
\usepackage{polyglossia}
\setdefaultlanguage{spanish}
\setotherlanguage{english}
\usepackage{enumerate}
\usepackage{lettrine}
\usepackage[table,xcdraw]{xcolor}
\usepackage[colorlinks=true,allcolors=negro]{hyperref}
\usepackage{steinmetz}
\usepackage{varioref}
\usepackage{fancyref}
\usepackage{cleveref}
\usepackage[colorinlistoftodos]{todonotes}
%\usepackage{showkeys}
\usepackage{textcomp}
\usepackage[siunitx, american]{circuitikz}
\usepackage{siunitx}
\usepackage{graphicx}
\usepackage{adjustbox}
\usepackage{multirow, array} 
\usepackage{float} 
\usepackage{booktabs}
\usepackage{verbatim}
\usepackage{tabulary}
\usepackage{multirow}
\usepackage{pgfplots}
\usepackage{nicefrac}
\usepackage{rotating}
\usepackage{caption}
\usepackage{cancel}
\usepackage{fancybox}
\usepackage{pstricks}
\usepackage{mdframed}
\usepackage{empheq}
\usepackage{mathtools}
\usepackage{xparse}
\usepackage{subfig}%Partición para Imágenes
\usetikzlibrary{mindmap, trees}
\usetikzlibrary{patterns}
\usetikzlibrary{shapes,arrows}
\newcommand{\grad}{\hspace{-2mm}$\phantom{a}^{\circ}$}

\newcommand{\fasorv}{\mathbb{V}}

%color##########################
\definecolor{gris}{rgb}{0.5,0.5,0.5}
\definecolor{morado}{RGB}{75,0,130}
\definecolor{VerdeOscuro}{RGB}{0,100,0}
\definecolor{AzulOscuro}{RGB}{0,0,139}
\definecolor{vinotinto}{RGB}{139,0,0}
\definecolor{magenta}{RGB}{199,21,133}
\definecolor{naranja}{RGB}{255,69,0}
\definecolor{oro}{RGB}{255,215,0}
\definecolor{uva}{RGB}{139,0,139}
\definecolor{ocre}{RGB}{139,69,19}
\definecolor{negro}{RGB}{0,0,0}
\definecolor{aguamarina}{RGB}{0,255,255}
\definecolor{AzulCielo}{RGB}{30,144,255}
\definecolor{verde}{RGB}{0,128,0}
\definecolor{miel}{RGB}{218,165,32}
\definecolor{crema}{RGB}{218,165,32}
\definecolor{PielBlanca}{RGB}{218,165,32}
\definecolor{chocolate}{RGB}{218,165,32}
\definecolor{AGris}{RGB}{119,136,153}
\definecolor{BGris}{RGB}{47,79,79}
\definecolor{cerceta}{RGB}{0,128,128}
\definecolor{blanco}{RGB}{255,255,255}
\definecolor{MarMedio}{RGB}{70,130,180}
\definecolor{azul}{RGB}{0,0,200}
\definecolor{rojo}{RGB}{200,0,0}
%%%%%%%%%%%%%%%%%%%%%%%%%%%%%%%%%%%%%%%%%%%%%%%%%%%%%%%%Salto de Línea en una celda de una Tabla%%%%%%%%%%%%%%%%%%%%%%%%%%%%%%%%%%%%%%%%%%%%%%%%%%%%%%%%
\newcommand{\specialcell}[2][c]{%
  \begin{tabular}[#1]{@{}c@{}}#2\end{tabular}}
%%%%%%%%%%%%%%%%%%%%%%%%%%%%%%%%%%%%%%%%%%%declaración para el uso de \abs{} como función de valor absoluto%%%%%%%%%%%%%%%%%%%%%%%%%%%%%%%%
\DeclarePairedDelimiter\abs{\lvert}{\rvert}%
\DeclarePairedDelimiter\norm{\lVert}{\rVert}%

% Swap the definition of \abs* and \norm*, so that \abs
% and \norm resizes the size of the brackets, and the 
% starred version does not.
\makeatletter
\let\oldabs\abs
\def\abs{\@ifstar{\oldabs}{\oldabs*}}
%
\let\oldnorm\norm
\def\norm{\@ifstar{\oldnorm}{\oldnorm*}}
\makeatother

\newcommand*{\Value}{\frac{1}{2}x^2}%
%%%%%%%%%%%%%%%%%%%%%%%%%%%%%%%%%%%%%%%%%%%%%%%%%%%%%%%%%%%%%%%%%%%%%%%%%%%%%%%%%%%%%%%%%%%%%%%%%%%%%%%%%%%%%%%%%%%%%%%%%%%%%%%%%%%%%%%%%%%%%%%%%%%%%%%%%%%%
%@@@@@@@@@@@@@@@@@@@@@@@@@@@@@@@@@@@@@@@@@@@@@@@@@@@@@@@@@@@@@@@@@@@@@@@@@@@@@@@@@@@@@@@@@@@@@ Declaracion para cajas de colores @@@@@@@@@@@@@@@@@@@@@@@@@@@@@@@@@@@@@@@@@@@@@@@@@@@@@@@@@@@@@@@@@@@@@@@@@@@@@@@@@@@@@@@@@@@@@@
\NewDocumentCommand{\framecolorbox}{oommm}
 {% #1 = width (optional)
  % #2 = inner alignment (optional)
  % #3 = frame color
  % #4 = background color
  % #5 = text
  \IfValueTF{#1}
   {%
    \IfValueTF{#2}
     {\fcolorbox{#3}{#4}{\makebox[#1][#2]{#5}}}
     {\fcolorbox{#3}{#4}{\makebox[#1]{#5}}}%
   }
   {\fcolorbox{#3}{#4}{#5}}%
 }
 %@@@@@@@@@@@@@@@@@@@@@@@@@@@@@@@@@@@@@@@@@@@@@@@@@@@@@@@@@@@@@@@@@@@@@@@@@@@@@@@@@@@@@@@@@@@@@@@@@@@@@@@@@@@@@@@@@@@@@@@@@@@@@@@@@@@@@@@@@@@@@@@@@@@@@@@@@@@@@@@@@@@@@@@@@@@@@@@@@@@@@@@@@@@@@@@@@@@@@@@@@@@@@@@@@@@@@@@@@@@@@@@@@@@@@@@@@@@@@@@@@@@@@@@@@@@@@@@@@@@@@@@@@@@@@@@@@@@@@@@@@@@@@@@@@@
 
 %€€€€€€€€€€€€€€€€€€€€€€€€€€€€€€€€€€€€€€€€€€€€€€€€€€€€€€€€€€€€€€€€ Declaración para fijar un sólo color al recuadro (caja) matemático \boxed €€€€€€€€€€€€€€€€€€€€€€€€€€€€€€€€€€€€€€€€€€€€€€€€€€€€€€€€€€€€€€€€€
 \newcommand{\boxia}{\boxed}
 \newcommand{\boxib}{\boxed}
 \newcommand{\boxic}{\boxed}
 \newcommand{\boxid}{\boxed}
 \newcommand{\boxie}{\boxed}
 \newcommand{\boxif}{\boxed}
 \newcommand{\boxig}{\boxed}
 \newcommand{\boxih}{\boxed}


% Naranja


\makeatletter
\renewcommand{\boxia}[1]{\textcolor{naranja}{%
  \fbox{\normalcolor\m@th$\displaystyle#1$}}}
\makeatother

% Verde

\makeatletter
\renewcommand{\boxib}[1]{\textcolor{verde}{%
		\fbox{\normalcolor\m@th$\displaystyle#1$}}}
\makeatother

% morado

\makeatletter
\renewcommand{\boxic}[1]{\textcolor{morado}{%
		\fbox{\normalcolor\m@th$\displaystyle#1$}}}
\makeatother

% vinotinto

\makeatletter
\renewcommand{\boxid}[1]{\textcolor{vinotinto}{%
		\fbox{\normalcolor\m@th$\displaystyle#1$}}}
\makeatother

% Cerceta

\makeatletter
\renewcommand{\boxie}[1]{\textcolor{cerceta}{%
		\fbox{\normalcolor\m@th$\displaystyle#1$}}}
\makeatother

% Aguamarina

\makeatletter
\renewcommand{\boxif}[1]{\textcolor{aguamarina}{%
		\fbox{\normalcolor\m@th$\displaystyle#1$}}}
\makeatother

% Azul

\makeatletter
\renewcommand{\boxig}[1]{\textcolor{azul}{%
		\fbox{\normalcolor\m@th$\displaystyle#1$}}}
\makeatother

% Azul Cielo

\makeatletter
\renewcommand{\boxih}[1]{\textcolor{AzulCielo}{%
		\fbox{\normalcolor\m@th$\displaystyle#1$}}}
\makeatother
%€€€€€€€€€€€€€€€€€€€€€€€€€€€€€€€€€€€€€€€€€€€€€€€€€€€€€€€€€€€€€€€€

%///////////////////////////////////////////
%			Algo más para circuitos
%///////////////////////////////////////////

\usetikzlibrary{arrows.meta}

%///////////////////////////////////////////
%///////////////////////////////////////////
%///////////////////////////////////////////

\newcommand{\oangle}{$^{\circ}$}
\newcommand{\oang}{^{\circ}}

% ---------------------- Función Seno, en español
\makeatletter
\appto\inlineextras@spanish{\renewcommand\sin{\qopname \relax m{sen}}}
\appto\blockextras@spanish{\renewcommand\sin{\qopname \relax m{sen}}}
\makeatother

%♛ ♛ ♛ ♛ ♛ ♛ ♛ ♛ ♛ ♛ ♛ ♛ ♛ ♛ ♛ ♛ ♛ ♛ ♛ ♛ ♛ ♛ ♛ ♛ ♛ ♛ ♛ ♛ ♛ ♛ ♛ ♛ ♛ ♛ ♛ ♛ ♛ ♛ ♛ ♛ ♛ ♛ Para el cruce de cables / Circuitikz♛ ♛ ♛ ♛ ♛ ♛ ♛ ♛ ♛ ♛ ♛ ♛ ♛ ♛ ♛ ♛ ♛ ♛ ♛ ♛ ♛ ♛ ♛ ♛ ♛ ♛ ♛ ♛ ♛ ♛ ♛ ♛ ♛ ♛ ♛ ♛ ♛ ♛ ♛ ♛ ♛ ♛ ♛ ♛ ♛ ♛ ♛ ♛ ♛ ♛ ♛ ♛ ♛ 


\tikzset{
  declare function={% in case of CVS which switches the arguments of atan2
    atan3(\a,\b)=ifthenelse(atan2(0,1)==90, atan2(\a,\b), atan2(\b,\a));},
  kinky cross radius/.initial=+.125cm,
  @kinky cross/.initial=+, kinky crosses/.is choice,
  kinky crosses/left/.style={@kinky cross=-},kinky crosses/right/.style={@kinky cross=+},
  kinky cross/.style args={(#1)--(#2)}{
    to path={
      let \p{@kc@}=($(\tikztotarget)-(\tikztostart)$),
          \n{@kc@}={atan3(\p{@kc@})+180} in
      -- ($(intersection of \tikztostart--{\tikztotarget} and #1--#2)!%
             \pgfkeysvalueof{/tikz/kinky cross radius}!(\tikztostart)$)
      arc [ radius     =\pgfkeysvalueof{/tikz/kinky cross radius},
            start angle=\n{@kc@},
            delta angle=\pgfkeysvalueof{/tikz/@kinky cross}180 ]
      -- (\tikztotarget)}}}
% ♛ ♛ ♛ ♛ ♛ ♛ ♛ ♛ ♛ ♛ ♛ ♛ ♛ ♛ ♛ ♛ ♛ ♛ ♛ ♛ ♛ ♛ ♛ ♛ ♛ ♛ ♛ ♛ ♛ ♛ ♛ ♛ ♛ ♛ ♛ ♛ ♛ ♛ ♛ ♛ ♛ ♛ ♛ ♛ ♛ ♛ ♛ ♛ ♛ ♛ ♛ ♛ ♛ ♛ ♛ ♛ ♛ ♛ ♛ ♛ ♛ ♛ ♛ ♛ ♛ ♛ ♛ ♛ ♛ ♛ ♛ ♛ ♛ ♛ ♛ ♛ ♛ ♛ ♛ ♛ ♛ ♛ ♛ ♛ ♛ ♛ ♛ ♛ ♛ ♛ ♛ ♛ ♛ ♛

%librería para matemáticas en Tikz
  \usetikzlibrary{math}

% ❅❀ ۞❅❀ ۞❅❀ ۞❅❀ ۞❅❀ ۞❅❀ ۞❅❀ ۞❅❀ ۞❅❀ ۞❅❀ ۞❅❀ ۞❅❀ ۞❅❀ ۞❅❀ ۞❅❀ ۞❅❀ ۞❅❀ ۞❅❀ ۞❅❀ ۞❅❀ ۞❅❀ ۞❅❀ ۞❅❀ ۞❅❀ ۞❅❀ ۞❅❀ ۞❅❀ ۞❅❀ ۞❅❀ ۞❅❀ ۞❅❀ ۞❅❀ ۞❅❀ ۞❅❀ ۞❅❀ ۞❅❀ ۞❅❀ ۞❅❀ ۞❅❀ ۞❅❀ ۞❅❀ ۞❅❀ ۞❅❀ ۞❅❀ ۞❅❀ ۞❅❀ ۞❅❀ ۞❅❀ ۞❅❀ ۞❅❀ ۞❅❀ ۞❅❀ ۞❅❀ ۞❅❀ ۞❅❀ ۞

%                            Empieza

% ❅❀ ۞❅❀ ۞❅❀ ۞❅❀ ۞❅❀ ۞❅❀ ۞❅❀ ۞❅❀ ۞❅❀ ۞❅❀ ۞❅❀ ۞❅❀ ۞❅❀ ۞❅❀ ۞❅❀ ۞❅❀ ۞❅❀ ۞❅❀ ۞❅❀ ۞❅❀ ۞❅❀ ۞❅❀ ۞❅❀ ۞❅❀ ۞❅❀ ۞❅❀ ۞❅❀ ۞❅❀ ۞❅❀ ۞❅❀ ۞❅❀ ۞❅❀ ۞❅❀ ۞❅❀ ۞❅❀ ۞❅❀ ۞❅❀ ۞❅❀ ۞❅❀ ۞❅❀ ۞❅❀ ۞❅❀ ۞❅❀ ۞❅❀ ۞❅❀ ۞❅❀ ۞❅❀ ۞❅❀ ۞❅❀ ۞❅❀ ۞❅❀ ۞❅❀ ۞❅❀ ۞❅❀ ۞❅❀ ۞


















\begin{document}


%--------------Título--------------------
\title{Formato IEEE}

%------%DATOS PERSONALES
\author{\emph{\small{Daniel Alejandro Rodríguez Chávez [25451381], daarodriguezch@unal.edu.co}}
\emph{\small{----------- [----------], \textit{-----------}}}}

%--------------fecha para modificar-------
\markboth{Informe Práctica N.~,~ \today}
{Shell \MakeLowercase{\textit{et al.}}: Informe de taller}

%------%RESUMEN Y PALABRAS CLAVE
                                                        
\maketitle
%%%%%%%%%%%%%%%%%%%%%%%%%%%%%%%%%%%%%%%%%%%%%%%%%%Abstract%%%%%%%%%%%%%%%%%%%%%%%%%%%%%%%%%%%%%%%%%%%%%%%%%%%%%
\renewcommand{\abstractname}{Abstract}
\begin{abstract}
%
%           Contenido
%
\end{abstract}
%%%%%%%%%%%%%%%%%%%%%%%%%%%%%%%%%%%%%%%%%%%%%%%%%%Key-Words%%%%%%%%%%%%%%%%%%%%%%%%%%%%%%%%%%%%%%%%%%%%%%%%%%%%
\begin{IEEEkeywords}
%
%Contenido
%
\end{IEEEkeywords}
%%%%%%%%%%%%%%%%%%%%%%%%%%%%%%%%%%%%%%%%%%%%%%%%%%%%Resumen%%%%%%%%%%%%%%%%%%%%%%%%%%%%%%%%%%%%%%%%%%%%%%%%%%%%%
\renewcommand{\abstractname}{Resumen}
\begin{abstract}
%
%           Contenido
%
\end{abstract}
%%%%%%%%%%%%%%%%%%%%%%%%%%%%%%%%%%%%%%%%%%%%%%%%%Palabras-Clave%%%%%%%%%%%%%%%%%%%%%%%%%%%%%%%%%%%%%%%%%%%%%%%%%%%%
\renewcommand\IEEEkeywordsname{Palabras Clave}
\begin{IEEEkeywords}
%
%           Contenido
%
\end{IEEEkeywords}





                                                    %MARCO TEÓRICO
%-----------------------------------------------------------------------------------------------------------------------
\section{Marco Teórico}

\subsection{}

\subsection{}


%-----------------------------------------------------------------------------------------------------------------------
                                                      
                                                    
                                                




                                                


  %PROCEDIMIENTO
%-----------------------------------------------------------------------------------------------------------------------
\section{Procedimiento}



\subsection {Primera parte}







\subsection {Segunda parte}




%-----------------------------------------------------------------------------------------------------------------------
%ANÁLISIS DE RESULTADOS
%-----------------------------------------------------------------------------------------------------------------------
\section{Análisis de Resultados}

%--------------------------tabla 1---------------------------------------








%-----------------------------------------------------------------------------------------------------------------------
%RESPUESTAS A LAS PREGUNTAS
%-----------------------------------------------------------------------------------------------------------------------

\section{Respuesta a las preguntas}

\begin{enumerate}
\item $\left< \textit{pregunta}\right>$  \\

\textit{Respuesta/.}

\item $\left< \textit{pregunta}\right>$\\

\textit{Respuesta/.}

\item $\left< \textit{pregunta}\right>$\\

\textit{Respuesta/.}

\item $\left< \textit{pregunta}\right>$\\

\textit{Respuesta/.}

\end{enumerate}
\input{Seccion5/Seccion5.tex}
%CONCLUSIÓNES (MÍNIMO 3)
%-----------------------------------------------------------------------------------------------------------------------
\section{Conclusiones}


\begin{itemize}
\item 

\item 

\item 

\item 
\end{itemize}





                                                    %MARCO TEÓRICO
%-----------------------------------------------------------------------------------------------------------------------
\section{Marco Teórico}

\subsection{}

\subsection{}


%-----------------------------------------------------------------------------------------------------------------------
                                                      
                                                    
                                                




                                                


%BIBLIOGRAFIA

\begin{thebibliography}{1}

\bibitem{cita1}\textsc{Nombre del Autor}, \textit{Sección del libro o documento},  Título del libro, número de edición , Editora, Fecha de edición, consultado el \today.

\bibitem{cita2}\textsc{Nombre del Autor}, \textit{Sección del libro o documento},  Título del libro, número de edición , Editora, Fecha de edición, consultado el \today.

\bibitem{cita3}\textsc{Nombre del Autor}, \textit{Sección del libro o documento},  Título del libro, número de edición , Editora, Fecha de edición, consultado el \today.


\end{thebibliography}

\end{document}
